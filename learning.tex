\documentclass[a4paper]{article}
\usepackage{xfrac} 
%\usepackage{enumitem} 
\usepackage{xcolor} 
\usepackage{graphicx}
\begin{document}
%\pagecolor{lightgray}
\begin{center}
\begin{large}
\begin{huge}

\textit{ \LaTeX Thought of the day} \\
%'\\' used for line break ;
\end{huge}

\end{large}
\end{center}

\LaTeX\ uses more $x^{3}$
iteration$ \Updownarrow $
and floating $ \frac{a}{b} $ v$\cdots$
such
$ x\sqrt{z} $ and the figures to make the document more \b x refined and polished.%\ In between is used for the spacing  
% here % command is used to write the comments which are ignored by the Latex and hence are not reflected in the output  
It is a document $\bigtriangledown system   $  $ y\oplus x$used for the publication of technical documents. Latex software not only saves time but also makes the text more attractive and refined. It is used by scientists, authors for the subjects such as mathematics, economics, psychology, engineering, etc. there are two types of editors available for the Latex, which are online and offline editors. It depends on the convenience and ease of the user to choose any particular editor for the Latex. The procedure to write the \lq \lq \textbf{cost makes the user concentrate} \rq \rq on the content instead of the format. It also has the feature of spell checking. 
b NOT a
boolean  1 image(cu)
$x \neq y$

\section{order listing: }

\begin{enumerate}
\item first
\item second 
\item third
\subsection{ordered sub listing: }
        \begin{enumerate}
          \item fourth 
          \item fifth 
         \end{enumerate}
\item fourth 
\item fifth 
\end{enumerate}

\subsection{unorder list: }
\begin{itemize}
\item sharif
\item omar 
         \begin{itemize}
         \item physics 
         \item chemistry 
         \end{itemize}
\item kazi 
\end{itemize}

\subsection*{Description order }
\begin{description}
 \item [sharif] a computer guy .
 \item [mathematics] mathematics is mother of science
\end{description}

\begin{flushright}
\large{kazi omar sharif} 
\end{flushright}
\textmd{medium} \newline
\noindent this is to 
certify 

\section{color: }
\textbf{The xcolor package supports adding colors to your text. Using this, you can set the \textcolor{red}{background}, font color, and the page background. You can choose \textcolor{green}{ from the} predefined colors or can create your color using RGB. The formulas of mathematics can also be colored.} 

\fcolorbox{red}{white} { this is color box} 
\begin{center}
\begin{figure}[b]



\includegraphics[width=.1\linewidth]{smile}\quad
\includegraphics[width=.1 \linewidth]{smile}\quad
\\[\baselineskip]
\includegraphics[width=.1\linewidth]{smile}\quad
\includegraphics[width=.1\linewidth]{smile}\quad

\caption{the picture of smile}





%\includegraphics[scale=1]{smile}
\includegraphics[scale=.1]{cse}
\end{figure}
\end{center}

\end{document}