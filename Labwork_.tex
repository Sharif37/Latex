\documentclass[12pt,a4paper]{article}
\usepackage{geometry}
\usepackage{coling2020}
\usepackage{times}
\usepackage{url}
\usepackage{datetime}
\usepackage{latexsym}
\usepackage{microtype}
\usepackage{amssymb}
\usepackage[cmex10]{amsmath}
\usepackage{amsmath,amssymb,amsfonts}
\usepackage{comment}
\usepackage{graphicx}
\graphicspath{{Figures/EPS/}{figures/}}
\usepackage[usestackEOL]{stackengine}
\usepackage{array, tabularx, booktabs, multicol, multirow, supertabular}
\usepackage{tabulary}
\usepackage{pdfpages}
\usepackage{color}
\usepackage[misc]{ifsym}
\usepackage{enumerate}
\usepackage[shortlabels]{enumitem}
\usepackage[linesnumbered,ruled,vlined]{algorithm2e}
\usepackage[ruled,vlined]{algorithm2e}
\usepackage{algorithmic}
\usepackage[hidelinks, bookmarks=false]{hyperref}
\usepackage{makecell}
\usepackage{calc}
\usepackage{times}
\usepackage{latexsym}
\usepackage{longtable}
\usepackage{authblk}
\usepackage{listings}
\hyphenation{an-aly-sis}
\hyphenation{an-aly-ses}
\colingfinalcopy 

\newcommand*{\affaddr}[1]{#1} % No op here. Customize it for different styles.
\newcommand*{\affmark}[1][*]{\textsuperscript{#1}}

\DeclareMathOperator*{\argmin}{\arg\!\min}
\DeclareMathOperator*{\argmax}{\arg\!\max}
 \graphicspath{ {./Latex/} }
\usepackage{subfiles}


\begin{document}
\subfile{Labwork}

\newpage
\section{ASSIGNMENT SOLUTIONS: }
{\tt{2.Display the names and hire dates for all employees who were hired before their managers, along 
with their manager’s names and hire dates.\\}}
\begin{lstlisting}[caption={Exercise-1},captionpos=,
backgroundcolor=\color{white},
language =SQL ,
breaklines=true,
frame=single ,
showspaces=false,
basicstyle=\ttfamily,
numbers=left ,
numberstyle=\tiny ,
rulecolor=\color{red},
keywordstyle=\color{blue},
commentstyle=\color{gray}]
select l.lastname ,l.hiredate ,m.lastname,m.hiredate
    from staffs l,staffs m
    where l.managerid=m.employeeid
    and l.hiredate<m.hiredate ;

\end{lstlisting} 

\vspace{1cm}


{\tt{3.Write a query to show the name of the department that has highest number of job categories. \\}}




\begin{lstlisting}[caption={Exercise-2},captionpos=,
backgroundcolor=\color{white},
language =SQL ,
breaklines=true,
frame=single ,
showspaces=false,
basicstyle=\ttfamily,
numbers=left ,
numberstyle=\tiny ,
rulecolor=\color{red},
keywordstyle=\color{blue},
commentstyle=\color{gray}]

select departmentname,departmentid
 from departments 
 where departmentid in
   (
   select departmentid 
   from staffs
   group by departmentid
   having count(*) in ( select max( numberofjob)
   from 
   ( select count(*) numberofjob
      from staffs 
   group by departmentid))) ;
\end{lstlisting} 




\vspace{1cm}
{\tt{4.Write a query who earn less than the average salary in their departments. Show lastName, 
departmentID, salary, and the average salary of that department. \\}}

\begin{lstlisting}[caption={Exercise-1},captionpos=,
backgroundcolor=\color{white},
language =SQL ,
breaklines=true,
frame=single ,
showspaces=false,
basicstyle=\ttfamily,
numbers=left ,
numberstyle=\tiny ,
rulecolor=\color{red},
keywordstyle=\color{blue},
commentstyle=\color{gray}]
select lastname,departmentid,salary
   from staffs s
   where salary < ( select avg(salary)
   from staffs t
   where s.departmentid=t.departmentid ) ;

\end{lstlisting} 




\vspace{1cm}
{\tt{5.Display the managerID and the salary of the highest paid employee for that manager. ** show the 
employees’s name as well\\}}


\begin{lstlisting}[caption={Exercise-1},captionpos=,
backgroundcolor=\color{white},
language =SQL ,
breaklines=true,
frame=single ,
showspaces=false,
basicstyle=\ttfamily,
numbers=left ,
numberstyle=\tiny ,
rulecolor=\color{red},
keywordstyle=\color{blue},
commentstyle=\color{gray}]
 
select managerid,salary ,lastname from staffs
  where salary in ( select max(salary) from staffs) ;

\end{lstlisting} 





\vspace{1cm}




\begin{lstlisting}[caption={Exercise-1},captionpos=,
backgroundcolor=\color{white},
language =SQL ,
breaklines=true,
frame=single ,
showspaces=false,
basicstyle=\ttfamily,
numbers=left ,
numberstyle=\tiny ,
rulecolor=\color{red},
keywordstyle=\color{blue},
commentstyle=\color{gray}]

 select departmentname,departmentid from departments
    natural join staffs
    where salary < any ( select sum(salary)*1/5 from staffs
   group by departmentid ) ;
 

\end{lstlisting} 



\vspace{1cm}
{\tt{7.Show the id of those employees who earn less than the average salary of the departments they 
work for.\\}}


\begin{lstlisting}[caption={Exercise-1},captionpos=,
backgroundcolor=\color{white},
language =SQL ,
breaklines=true,
frame=single ,
showspaces=false,
basicstyle=\ttfamily,
numbers=left ,
numberstyle=\tiny ,
rulecolor=\color{red},
keywordstyle=\color{blue},
commentstyle=\color{gray}]
 
  select departmentid
   from staffs s
   where salary< (select avg(salary) from staffs t
    where s.departmentid=t.departmentid) ;

\end{lstlisting} 




\vspace{1cm}
{\tt{8.Who is the oldest employee in the company? Show his current position, salary, working 
department and his immediate supervisor.
\\}}


\begin{lstlisting}[caption={Exercise-1},captionpos=,
backgroundcolor=\color{white},
language =SQL ,
breaklines=true,
frame=single ,
showspaces=false,
basicstyle=\ttfamily,
numbers=left ,
numberstyle=\tiny ,
rulecolor=\color{red},
keywordstyle=\color{blue},
commentstyle=\color{gray}]
 
 create table jobtime1
   as
   select lastname ,jobid,managerid,round((sysdate-hiredate),0) as time
   from staffs ; 
   
   select * from jobtime1
    order by time desc
    fetch first 1 rows only ;
 

\end{lstlisting} 














\end{document}
