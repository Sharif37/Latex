\documentclass[12pt,a4paper]{article}
\usepackage{geometry}
\usepackage{coling2020}
\usepackage{times}
\usepackage{url}
\usepackage[table,xcdraw]{xcolor}
\usepackage{datetime}
\usepackage{latexsym}
\usepackage{microtype}
\usepackage{amssymb}
\usepackage[cmex10]{amsmath}
\usepackage{amsmath,amssymb,amsfonts}
\usepackage{comment}
\usepackage{graphicx}
\usepackage{float}
\graphicspath{{Figures/EPS/}{figures/}}
\usepackage[usestackEOL]{stackengine}
\usepackage{array, tabularx, booktabs, multicol, multirow, supertabular}
\usepackage{tabulary}
\usepackage{pdfpages}
\usepackage{color}
\usepackage[misc]{ifsym}
\usepackage{enumerate}
\usepackage[shortlabels]{enumitem}
\usepackage[linesnumbered,ruled,vlined]{algorithm2e}
\usepackage[ruled,vlined]{algorithm2e}
\usepackage{algorithmic}
\usepackage[hidelinks, bookmarks=false]{hyperref}
\usepackage{makecell}
\usepackage{calc}
\usepackage{times}
\usepackage{latexsym}
\usepackage{longtable}
\usepackage{authblk}

\hyphenation{an-aly-sis}
\hyphenation{an-aly-ses}
\colingfinalcopy 

\newcommand*{\affaddr}[1]{#1} % No op here. Customize it for different styles.
\newcommand*{\affmark}[1][*]{\textsuperscript{#1}}

\DeclareMathOperator*{\argmin}{\arg\!\min}
\DeclareMathOperator*{\argmax}{\arg\!\max}
\graphicspath{ {./Latex/} }

\begin{document}
\begin{center}
\textbf{Quick sort applies  divide \& conquer paradigm. }\\
\end{center}

%Alogirthm 
\begin{algorithm}
\caption{QUICKSORT ALGORITHM: }
\vspace*{.2 cm}
\begin{algorithmic}

\STATE QUICKSORT(A,p,r)
\IF { p < r}
\STATE q= PARTITION(A,p,r)
\STATE QUICKSORT(A,p,q-1)
\STATE QUICKSORT(A,q+1,r) 
\ENDIF
\end{algorithmic}
\end{algorithm}


\vspace{1 cm}

\begin{algorithm}
\caption{Partition algorithm: } 
\begin{algorithmic}
\STATE PARTITION(A,p,r)\\
\STATE x=A[r]\\
\STATE i=p-1 \\
\FOR {j=p to r-1}

\IF{ A[r] $ \leq $ x} 
\STATE i=i+1 
\ENDIF
\STATE exchange A[i] with A[j]
\ENDFOR
\STATE exchange A[i+1] with A[r]

\RETURN i+1 
 
\end{algorithmic}

\end{algorithm}


pen \& paper description of quick sort\:


%create table 

\begin{table}[H]
\begin{tabular}{|l|l|l|l|l|l|}
\hline
5 & 2 & 4 & 6 & 1 & 3 \\ \hline
\end{tabular}
\end{table}



%analysics






























































\end{document}
